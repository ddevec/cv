%%%%%%%%%%%%%%%%%%%%%%%%%%%%%%%%%%%%%%%%%
% Medium Length Graduate Curriculum Vitae
% LaTeX Template
% Version 1.1 (9/12/12)
%
% This template has been downloaded from:
% http://www.LaTeXTemplates.com
%
% Original author:
% Rensselaer Polytechnic Institute (http://www.rpi.edu/dept/arc/training/latex/resumes/)
%
% Important note:
% This template requires the res.cls file to be in the same directory as the
% .tex file. The res.cls file provides the resume style used for structuring the
% document.
%
%%%%%%%%%%%%%%%%%%%%%%%%%%%%%%%%%%%%%%%%%

%----------------------------------------------------------------------------------------
%	PACKAGES AND OTHER DOCUMENT CONFIGURATIONS
%----------------------------------------------------------------------------------------
%

\documentclass[margin, 10pt]{res} % Use the res.cls style, the font size can be changed to 11pt or 12pt here

\usepackage{helvet} % Default font is the helvetica postscript font
%\usepackage{newcent} % To change the default font to the new century schoolbook postscript font uncomment this line and comment the one above
%


%\titlespacing\subsection{0pt}{12pt plus 4pt minus 2pt}{0pt plus 2pt minus 2pt}
%\titlespacing\subsubsection{0pt}{12pt plus 4pt minus 2pt}{0pt plus 2pt minus 2pt}


\setlength{\textwidth}{5.1in} % Text width of the document

\begin{document}

%----------------------------------------------------------------------------------------
%	Name And Address Section
%----------------------------------------------------------------------------------------

\moveleft.5\hoffset\centerline{\large\bf David Devecsery} % Your name at the top
 
\moveleft\hoffset\vbox{\hrule width\resumewidth height 1pt}\smallskip % Horizontal line after name; adjust line thickness by changing the '1pt'
 
\moveleft.5\hoffset\centerline{University of Michigan - Ann Arbor} % Your address
\moveleft.5\hoffset\centerline{4929 BBB, 2260 Hayward St} % Your address
\moveleft.5\hoffset\centerline{Ann Arbor, MI 48109}
\moveleft.5\hoffset\centerline{ddevec@umich.edu}
\moveleft.5\hoffset\centerline{http://www.umich.edu/\textasciitilde{}devec}

%----------------------------------------------------------------------------------------

\begin{resume}
 

\section{Education}

{\sl PhD. Candidate,} \\
University of Michigan, Ann Arbor, MI, Expected graduation April 2017 \\
Concentration: Computer Science Engineering \\
Advisor: Peter M. Chen

{\sl Masters of Science,} Engineering \\
University of Michigan, Ann Arbor, MI, April 2013 \\
Concentration: Computer Science Engineering

{\sl Bachelor of Science,} Engineering \\
University of Michigan, Ann Arbor, MI, April 2011 \\
Concentration: Computer Engineering


\section{Research \\Area}
Today computers by default discard their rich history of prior computation and
memory states.  My vision is to create computer systems with the default of
remembering all past computation and furthermore making it practical to query
the massive, rich database of computation recorded.  My research focuses on
creating system-level abstractions and developing novel analysis methodologies
to accomplish this goal.  

\section{Research Projects}
%\begin{itemize}
%\item 
\textbf{Optimistic Hybrid Analysis} -- A novel analysis methodology
which combines speculation, static analysis, and dynamic analysis to
ultimately produce a dynamic analysis which is far more efficient than
a traditional dynamic analysis, but just as accurate (under
submission).

%\item
\textbf{Eidetic Systems} -- A computer operating system which by
default remembers and can query all computation (process memory states, and
inter-process communication) done on it.

%\item
\textbf{Parallel Analysis} -- I've worked on parallelizing
\textbf{Data Race Detection} and \textbf{Dynamic Information Flow
Tracking}.
%\end{itemize}

%Broadly, my research focus is on Operating Systems.  More specifically
%my research focuses on improving the reliability of programs through
%system-level abstractions.  My most recent work is about ``Eidetic
%Systems," or systems which store their entire computation history.
%``Eidetic Systems" can analyze any computation done on them, to
%understand how past computations have effected current states.

\section{Non-Research \\ Projects}
\textbf{TalkingBook} --
A computer system catered to the challenges of the developing world,
specifically targeting extremely low-cost and power consumption.  We
created a custom microprocessor with novel memory architecture and
operating system designed to be accessible to the world's
poorest people.  A joint work between the University of Michigan and
Literacy Bridge (http://literacybridge.org).

\section{Publications}
\textbf{Peer-Reviewed Publications}\\
Andrew Quinn, David Devecsery, Peter M. Chen, and Jason Flinn,
``JetStream: Cluster-Scale Parallelization of Information Flow Queries"
{\sl to appear in Proceedings of the 12th USENIX Symposium on Operating
Systems Design and Implementation} (OSDI), November 2016.

David Devecsery, Michael Chow, Xianzheng Dou, Peter M Chen, Jason Flinn,
``Eidetic Systems" {\sl Proceedings of the 2014 conference on Operating
System Design and Implementation} (OSDI), October 2014.

Benjamin Wester, David Devecsery, Peter M. Chen, Jason Flinn, Satish
Narayanasamy, ``Parallelizing Data Race Detection", {\sl Proceedings of the
2013 International Conference on Architectural Support for Programming
Languages and Operating Systems} (ASPLOS), March 2013.

Foo, Z.; Devecsery, D.; Ghaed, M.; Lee, I.; Madhavan, A.; Park, Y.;
Rao, A.; Renner, Z.; Roberts, N.; Schulman, A.; Vinay, V.; Wieckowski,
M.; Yoon, D.; Schmidt, C.; Schmid, T.; Dutta, P.; Chen, P.; Blaauw,
D.; , ``A low-cost audio computer for information dissemination among
illiterate people groups," {\sl Custom Integrated Circuits Conference}
(CICC), 2012 IEEE , vol., no., pp.1-4, 9-12 Sept. 2012

Huey-Ming Tzeng, Atul Prakash, Mark Brehob, Allison Anderson, David
Andrew Devecsery and Chang-Yi Yin, ``How Feasible Was a Bed-Height
Alert System?”, {\sl Clinical Nursing Research} (SCI, SSCI) 2012

Huey-Ming Tzeng, Atul Prakash, Mark Brehob, Allison Anderson, David
Andrew Devecsery and Chang-Yi Yin, ``Keeping patient beds in a low
position: an exploratory descriptive study to continuously monitor the
height of patient beds in an adult acute surgical inpatient care
setting," {\sl Contemporary Nursing} June 2012.

Zhiyoong Foo, David Devecsery, Thomas Schmid, N. Clark, Rebecca Frank,
M. Ghaed, Y. Kuo, In Hee Lee, Youn Sung Park, Zachary Renner,
Nathaniel Slottow, Vikas Vinay, Michael Wieckowski, Dongmin Yoon,
David Blaauw, Peter M. Chen, Prabal Dutta, Cliff Schmidt, ``A Case for
Custom Silicon in Enabling Low-Cost Information Technology for
Developing Regions" , {\sl Proceedings of the 2010 Symposium on Computing
for Development} (SIGDEV) , December 2010.

\textbf{Presentations}\\
``Eidetic Systems" David Devecsery, Michael Chow, Xianzheng Dou,
Peter M Chen, Jason Flinn, {\sl Proceedings of the 2014 conference
on Operating System Design and Implementation} (OSDI), October 2014.

``A Case for Custom Silicon in Enabling Low-Cost Information
Technology for Developing Regions" Zhiyoong Foo, David Devecsery,
Thomas Schmid, N. Clark, Rebecca Frank, M. Ghaed, Y. Kuo, In Hee Lee,
Youn Sung Park, Zachary Renner, Nathaniel Slottow, Vikas Vinay,
Michael Wieckowski, Dongmin Yoon, David Blaauw, Peter M. Chen, Prabal
Dutta, Cliff Schmidt,  , {\sl Proceedings of the 2010 Symposium on
Computing for Development} (SIGDEV) , December 2010. (Presentation by
David Devecsery)

\textbf{Posters}\\
``Optimistic Static Analysis" David Devecsery, Peter M Chen, Jason
Flinn, Satish Narayanasamy, {\sl The 24th ACM Symposium on Operating Systems
Principals} (SOSP), October 2015.

``Eidetic Systems" David Devecsery, Michael Chow, Xianzheng Dou,
Peter M Chen, Jason Flinn, {\sl The 2014 conference
on Operating System Design and Implementation} (OSDI), October 2014.

``Flexible Replay Groups" David Devecsery, Peter M Chen, Jason Flinn,
{\sl The 24th ACM Symposium on Operating Systems
Principals} (SOSP), November 2013.

\section{Teaching}
\textbf{TA for Introduction to Operating Systems (EECS 482)} \hfill Fall 2014 \\
\textbf{TA for Computer Architecture (EECS 470)} \hfill Winter 2011 \\

%
%\section{Representative \\ Course \\ Projects}
%\textbf{Advanced Security} \hfill Fall '12
%\begin{itemize}
%\item Designed a system to provide efficient unlimited watchpoints
%	using AMD's proposed transactional memory specification.
%\end{itemize}
%
%\textbf{Advanced Compilers} \hfill Winter '11
%\begin{itemize}
%\item Created an efficient Cache Profiler using Intel performance
%	counters, and Precise Event Based Sampling
%\end{itemize}
%
%\textbf{Advanced Operating Systems} \hfill Fall '11
%\begin{itemize}
%\item Tested transient failures of NAND Flash, and flash translation
%	layer vulnerabilities to these failures.
%\end{itemize}

\section{Service}
CSEG Board Game Night Organizer \hfill Winter 2013-Fall 2014 \\
Software Reading Group Organizer \hfill Winter 2012-Winter 2013 \\
Software Reading Group Volunteer Speaker \hfill Fall 2011-Present \\
Tutor in Computer Architecture and Operating Systems\hfill Fall 2011 \\
Mentor for TalkingBook Project \hfill Fall '12 - Present \\
%Eagle Scout \\

\end{resume}
\end{document}
